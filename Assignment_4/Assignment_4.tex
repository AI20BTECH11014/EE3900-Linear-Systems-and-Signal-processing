\documentclass[journal,12pt,twocolumn]{IEEEtran}

\usepackage{setspace}
\usepackage{gensymb}
\singlespacing
\usepackage[cmex10]{amsmath}

\usepackage{amsthm}

\usepackage{mathrsfs}
\usepackage{txfonts}
\usepackage{stfloats}
\usepackage{bm}
\usepackage{cite}
\usepackage{cases}
\usepackage{subfig}

\usepackage{longtable}
\usepackage{multirow}

\usepackage{enumitem}
\usepackage{mathtools}
\usepackage{steinmetz}
\usepackage{tikz}
\usepackage{circuitikz}
\usepackage{verbatim}
\usepackage{tfrupee}
\usepackage[breaklinks=true]{hyperref}
\usepackage{graphicx}
\usepackage{tkz-euclide}

\usetikzlibrary{calc,math}
\usepackage{listings}
    \usepackage{color}                                            %%
    \usepackage{array}                                            %%
    \usepackage{longtable}                                        %%
    \usepackage{calc}                                             %%
    \usepackage{multirow}                                         %%
    \usepackage{hhline}                                           %%
    \usepackage{ifthen}                                           %%
    \usepackage{lscape}     
\usepackage{multicol}
\usepackage{chngcntr}

\DeclareMathOperator*{\Res}{Res}

\renewcommand\thesection{\arabic{section}}
\renewcommand\thesubsection{\thesection.\arabic{subsection}}
\renewcommand\thesubsubsection{\thesubsection.\arabic{subsubsection}}

\renewcommand\thesectiondis{\arabic{section}}
\renewcommand\thesubsectiondis{\thesectiondis.\arabic{subsection}}
\renewcommand\thesubsubsectiondis{\thesubsectiondis.\arabic{subsubsection}}


\hyphenation{op-tical net-works semi-conduc-tor}
\def\inputGnumericTable{}                                 %%

\lstset{
%language=C,
frame=single, 
breaklines=true,
columns=fullflexible
}
\begin{document}

\newcommand{\BEQA}{\begin{eqnarray}}
\newcommand{\EEQA}{\end{eqnarray}}
\newcommand{\define}{\stackrel{\triangle}{=}}
\bibliographystyle{IEEEtran}
\raggedbottom
\setlength{\parindent}{0pt}
\providecommand{\mbf}{\mathbf}
\providecommand{\pr}[1]{\ensuremath{\Pr\left(#1\right)}}
\providecommand{\qfunc}[1]{\ensuremath{Q\left(#1\right)}}
\providecommand{\sbrak}[1]{\ensuremath{{}\left[#1\right]}}
\providecommand{\lsbrak}[1]{\ensuremath{{}\left[#1\right.}}
\providecommand{\rsbrak}[1]{\ensuremath{{}\left.#1\right]}}
\providecommand{\brak}[1]{\ensuremath{\left(#1\right)}}
\providecommand{\lbrak}[1]{\ensuremath{\left(#1\right.}}
\providecommand{\rbrak}[1]{\ensuremath{\left.#1\right)}}
\providecommand{\cbrak}[1]{\ensuremath{\left\{#1\right\}}}
\providecommand{\lcbrak}[1]{\ensuremath{\left\{#1\right.}}
\providecommand{\rcbrak}[1]{\ensuremath{\left.#1\right\}}}
\theoremstyle{remark}
\newtheorem{rem}{Remark}
\newcommand{\sgn}{\mathop{\mathrm{sgn}}}
\providecommand{\abs}[1]{\vert#1\vert}
\providecommand{\res}[1]{\Res\displaylimits_{#1}} 
\providecommand{\norm}[1]{\lVert#1\rVert}
%\providecommand{\norm}[1]{\lVert#1\rVert}
\providecommand{\mtx}[1]{\mathbf{#1}}
\providecommand{\mean}[1]{E[ #1 ]}
\providecommand{\fourier}{\overset{\mathcal{F}}{ \rightleftharpoons}}
%\providecommand{\hilbert}{\overset{\mathcal{H}}{ \rightleftharpoons}}
\providecommand{\system}{\overset{\mathcal{H}}{ \longleftrightarrow}}
	%\newcommand{\solution}[2]{\textbf{Solution:}{#1}}
\newcommand{\solution}{\noindent \textbf{Solution: }}
\newcommand{\cosec}{\,\text{cosec}\,}
\providecommand{\dec}[2]{\ensuremath{\overset{#1}{\underset{#2}{\gtrless}}}}
\newcommand{\myvec}[1]{\ensuremath{\begin{pmatrix}#1\end{pmatrix}}}
\newcommand{\mydet}[1]{\ensuremath{\begin{vmatrix}#1\end{vmatrix}}}
\numberwithin{equation}{subsection}
\makeatletter
\@addtoreset{figure}{problem}
\makeatother
\let\StandardTheFigure\thefigure
\let\vec\mathbf
\renewcommand{\thefigure}{\theproblem}
\def\putbox#1#2#3{\makebox[0in][l]{\makebox[#1][l]{}\raisebox{\baselineskip}[0in][0in]{\raisebox{#2}[0in][0in]{#3}}}}
     \def\rightbox#1{\makebox[0in][r]{#1}}
     \def\centbox#1{\makebox[0in]{#1}}
     \def\topbox#1{\raisebox{-\baselineskip}[0in][0in]{#1}}
     \def\midbox#1{\raisebox{-0.5\baselineskip}[0in][0in]{#1}}
\vspace{3cm}
\title{ Assignment-4}
\author{Manikanta vallepu - AI20BTECH11014}
\maketitle
\newpage
\bigskip
\renewcommand{\thefigure}{\theenumi}
\renewcommand{\thetable}{\theenumi}
\newcommand{\R}{\mathbb{R}}
Download all  latex-tikz codes from 
\begin{lstlisting}
https://github.com/AI20BTECH11014/EE3900-Linear-Systems-and-Signal-processing/blob/main/Assignment_4/Assignment_4.tex
\end{lstlisting}
\vspace{0.5cm}
\section{QUESTION: Linear Forms Q.2.14}
The sum of the perpendicular distances of a variable point $\vec{P}$ from the lines 
$$\myvec{1&1} \vec{x} = 0$$
$$\myvec{3&-2} \vec{x} = -7$$
is always 10. Show that $\vec{P}$ must move on a line.

\section{SOLUTION}
The foot of perpendicular from point $\vec{P}$ to line $ \vec{n}^{\top }\vec{x} =c$ is given as $\vec{P} + \alpha \vec{n}$,where $\alpha \in \R$
\begin{align}
\vec{n}^{\top }\brak{\vec{P} + \alpha\vec{n}} = c\\
 \vec{n}^{\top }\vec{P} + \norm{\vec{n}}^{2}\alpha =c \label{1}
\end{align}
The perpendicular distance from a point $\vec{P}$ to line $ \vec{n}^{\top }\vec{x} =c$ is given as,
\begin{align}
&=\norm{\vec{P} - \brak{\vec{P} + \alpha\vec{n}}}\\
&=\abs{\alpha}\norm{\vec{n}} \label{2}
\end{align}
using \eqref{1} in \eqref{2}, perpendicular distance is given as,
\begin{align}
&=\dfrac{\abs{c- \vec{n}^{\top }\vec{P}}}{\norm{\vec{n}}} \label{3}
\end{align}
The sum of the perpendicular distances of a variable point $\vec{P}$ from the lines 
$$\vec{n_{1}}^{\top } \vec{x} = c_{1}$$
$$\vec{n_{2}}^{\top } \vec{x} = c_{2}$$
is always d,then,
\begin{align}
\dfrac{\abs{c_{1}- \vec{n_{1}}^{\top }\vec{P}}}{\norm{\vec{n_{1}}}}+ \dfrac{\abs{c_{2}- \vec{n_{2}}^{\top }\vec{P}}}{\norm{\vec{n_{2}}}} = d
\end{align}
given,
\begin{align}
\dfrac{\abs{0- \myvec{1&1} \vec{P}}}{\norm{\myvec{1&1}}} + \dfrac{\abs{7+ \myvec{3&-2} \vec{P}}}{\norm{\myvec{3&-2}}}=10\\
\dfrac{1}{\sqrt{2}} \abs{\myvec{1&1}\vec{P}} + \dfrac{1}{\sqrt{13}} \abs{\myvec{3&-2}\vec{P} +7} = 10
\end{align}
$\therefore$ point $\vec{P}$ lies on either of the lines
\begin{align}
L_1&:\myvec{\dfrac{1}{\sqrt{2}}+\dfrac{3}{\sqrt{13}}& \dfrac{1}{\sqrt{2}}-\dfrac{2}{\sqrt{13}}}\vec{P} = 10 - \dfrac{7}{\sqrt{13}}\\
L_2&:\myvec{\dfrac{1}{\sqrt{2}}-\dfrac{3}{\sqrt{13}}& \dfrac{1}{\sqrt{2}}+\dfrac{2}{\sqrt{13}}}\vec{P} = 10 + \dfrac{7}{\sqrt{13}}\\
L_3&:\myvec{\dfrac{1}{\sqrt{2}}+\dfrac{3}{\sqrt{13}}& \dfrac{1}{\sqrt{2}}-\dfrac{2}{\sqrt{13}}}\vec{P} = -10 - \dfrac{7}{\sqrt{13}}\\
L_4&:\myvec{-\dfrac{1}{\sqrt{2}}+\dfrac{3}{\sqrt{13}}& -\dfrac{1}{\sqrt{2}}-\dfrac{2}{\sqrt{13}}}\vec{P} = 10 - \dfrac{7}{\sqrt{13}}
\end{align}
\end{document}
